\documentclass{article}
\usepackage[utf8]{inputenc}
\usepackage[russian]{babel}           
\usepackage[justification=centering]{caption}
\usepackage[backend=bibtex,sorting=none]{biblatex}
\usepackage{fontspec}
\usepackage[final]{graphicx}
\usepackage{tabularx,booktabs}
\newcolumntype{C}{>{\centering\arraybackslash}X} % centered version of "X" type
\setlength{\extrarowheight}{3pt}
\usepackage{float}
\usepackage{subcaption}
\usepackage{mathtools}
\usepackage{adjustbox}
\usepackage{listings}
\usepackage{array}
\usepackage{xcolor}
\usepackage{graphicx}
\usepackage{tikz}
\usepackage{pgfplots}
\usepackage{fontspec}
\usepackage{amssymb}
%\usepackage{unicode-math}
\graphicspath{{./images/}}


\definecolor{codegreen}{rgb}{0,0.6,0}
\definecolor{codegray}{rgb}{0.5,0.5,0.5}
\definecolor{codepurple}{rgb}{0.58,0,0.82}
\definecolor{backcolour}{rgb}{0.95,0.95,0.92}

\newcommand{\A}{\underline{A}}
\newcommand{\B}{\underline{B}}
\newcommand{\und}[1]{\underline{#1}}
\newcommand{\sqft}[3]{#1_{#2}, \ldots, #1_{#3}}

\usepackage[%
        a4paper,%
        includehead,%
        left=2cm,%
        top=0.95cm,%
        right=2cm,%
        bottom=1.65cm,%
        headheight=0.7cm,%
        headsep=0.3cm,%
        footskip=0.8cm]{geometry}
\special{papersize=210mm,297mm}

\lstdefinestyle{mystyle}{
    language=xml,
    backgroundcolor=\color{white},   
    commentstyle=\color{codegreen},
    keywordstyle=[1]{\color{magenta}},
    keywordstyle=[2]{\color{blue}},
    numberstyle=\color{codegray},
    stringstyle=\color{codepurple},
    basicstyle=\ttfamily\footnotesize,
    keywords=[1]{*,<,>,algo, params, param, block, vertex, arg, in},            % if you want to add more keywords to the set
    keywords=[2]{*, id, name, condition, dims, type, val, bsrc, src, value},            % if you want to add more keywords to the set
    breakatwhitespace=false,         
    breaklines=true,                 
    captionpos=b,                    
    keepspaces=true,                 
    numbers=left,                    
    numbersep=10pt,
    xleftmargin=7mm,
    xrightmargin=0mm,
    showspaces=false,                
    showstringspaces=false,
    showtabs=false,                  
    tabsize=4
}
\lstset{style=mystyle}
\lstset{linewidth=1.1\linewidth}
\setmainfont{Times New Roman}
\setmonofont{Monaco}
\title{Отчёт по практическому заданию (2) в рамках курса\\«Суперкомпьютерное моделирование и технологии»\\Вариант 5}
\author{Никифоров Никита Игоревич, гр. 621\\nickiforov.nik@gmail.com}
\date{Октябрь 2022}
\pgfplotsset{compat=1.17}
            
\renewcommand{\baselinestretch}{1.5}
\begin{document}
\maketitle
\newpage
\section{Задача}
    Необходимо реализовать численный метод Монте-Карло нахождения значения интеграла
    в заданной области. Для реализации метода предлагается использовать языки программирования 
    {\tt C/C++}, с использованием библиотеки параллельного вычисления {\tt MPI}.

    Необходимо провести исследование реализованного численного метода для
    заданного интеграла, области и точности 
    на параллельных вычислительных системах ВМК МГУ: {\tt IBM Blue Gene/P}, {\tt IBM Polus}
\section{Математическая постановка задачи}
    Пусть функция \(f(x, y, z) = x^3*y^2*z\)~--- непрерывна в ограниченной замкнутой области \(G \subset \mathbb{R}^3\).
    Требуется вычислить определённый интеграл:
    \begin{equation}
        I\ =\ \iiint_G f(x, y, z)\ dxdydz = \iiint_G x^3*y^2*z\ dxdydz,
        \label{eq:integ}
    \end{equation}
    где область \(G={(x,y,z):\ −1 \leq x \leq 0,\ −1\leq y \leq 0,\ −1\leq z \leq 0}\)
\section{Численный метод решения задачи (Монте-Карло)}
    Пусть заданная область \(G\) ограниченна параллелепипедом \(P:\ a_1 \leq x \leq b_2,\ a_2 \leq y \leq b_2,\ a_3 \leq z \leq b_3\)
    Рассмотрим функцию определённую на параллелепипеде \(P\):
    \begin{equation}
        F(x, y, z)\ =\
        \begin{cases}
            f(x, y, z), & (x, y, z) \in G\\
            0         , & (x, y, z) \notin G\\
        \end{cases}
        \label{eq:funcF}
    \end{equation}

    Преобразуем искомый интеграл~\ref{eq:integ} - подставив функцию~\ref{eq:funcF}.
    
    \begin{equation}
        I\ =\ \iiint_G f(x, y, z)\ dxdydz\ =\ \iiint_P F(x, y, z)\ dxdydz,
        \label{eq:integF}
    \end{equation}

    Пусть \(p_1:\ (x_1, y_1, z_1),\ p_2:\ (x_2, y_2, z_2),\ldots, p_n:\ (x_n, y_n, z_n)\)~---
    случайные точки равномерно распределённые по области \(P\). Тогда в качестве приближённого
    значения интеграла~\ref{eq:integF} предлагается использовать выражение:
    \begin{equation}
        I \approx |P| * \frac{1}{n} *\sum_{i=1}^n F(p_i),
    \end{equation}
    где \(|P|\) — объём параллелепипеда \(P\), \(|P| = (b_1 − a_1)(b_2 − a_2)(b_3 − a_3)\).
\section{Аналитическое решение задачи}
    Найдём аналитически интеграл~\ref{eq:integ}:
    \begin{equation}
        \begin{aligned}
        I &= \iiint_G x^3*y^2*z\ dxdydz\ = \int_{-1}^{0} dx\int_{-1}^{0} dy \int_{-1}^{0} x^3*y^2*z\ dz\ =\\
          &\frac{x^4}{4}\biggr\rvert_{-1}^{0}\frac{y^3}{3}\biggr\rvert_{-1}^{0}\frac{z^2}{2}\biggr\rvert_{-1}^{0}\ =\ 
            \frac{1}{24} * x^4y^3z^2\biggr\rvert_{-1}^{0}\ =\ \frac{1}{24} \approx 0.0416(6)
        \end{aligned}
    \end{equation}
\section{Программная реализация}
\end{document}
